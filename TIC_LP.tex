\documentclass[a4paper,10pt]{article}

% Paquetes requeridos
\usepackage[utf8]{inputenc}
\usepackage[spanish]{babel}
\usepackage{csquotes}
\usepackage{amsmath, amssymb, amsfonts}
\usepackage{graphicx}
\usepackage[style=apa, backend=biber, natbib=true, language=spanish, url=true]{biblatex}
\usepackage{tocloft} % Para personalizar el índice
\usepackage[left=3.5cm,right=2.5cm,top=3.5cm,bottom=3.8cm]{geometry}
\usepackage{setspace} % Espaciado
\usepackage{titlesec} % Para personalizar los títulos
\usepackage{fancyhdr} % Para personalizar encabezados y pies de página
\usepackage{newtxtext}

\pagestyle{fancy}
\fancyhf{} % Limpia encabezados y pies de página
\renewcommand{\headrulewidth}{0pt} % Elimina la línea del encabezado

\addbibresource{referencias.bib}
\DeclareLanguageMapping{spanish}{spanish-apa}
% Configuraciones
\setlength{\parskip}{6pt} % Espacio entre párrafos
\setstretch{1.15} % Espacio entre líneas

\renewcommand{\cftsecleader}{\cftdotfill{\cftdotsep}} % Para puntos en el índice

% Estilos para títulos y subtítulos
\titleformat{\section}
{\normalfont\fontsize{12}{15}\bfseries}{\thesection}{1em}{}
\titleformat{\subsection}
{\normalfont\fontsize{10}{13}\bfseries}{\thesubsection}{1em}{}
\titleformat{\subsubsection}
{\normalfont\fontsize{10.5}{13}\bfseries}{\thesubsubsection}{1em}{}

\usepackage[hypertexnames=false, colorlinks=true, 
linkcolor=blue, 
citecolor=blue, 
urlcolor=blue, 
linkbordercolor={1 1 0}, 
citebordercolor={1 1 0}, 
urlbordercolor={1 1 0}, 
filecolor=blue, 
pdfborderstyle={/S/U/W 1}]{hyperref}

% Inicio del documento
\begin{document}
	\pagestyle{empty}
	% Carátula
	\begin{titlepage}
		\centering
		\vspace*{1.5cm}
		\includegraphics[width=0.3\textwidth]{unerlogo.png}
		\linebreak
		{\fontsize{14}{17}\bfseries Título del Documento\par}
		{\small [Nombre del primer alumno]\par}
		{\small [Nombre del segundo alumno]\par}
		{\normalsize [Nombre de la Universidad]\par}
		{\normalsize Facultad de [nombre de la facultad]\par}
		{\normalsize Carrera de [nombre de la carrera]\par}
		{\small Correo electrónico de autor\par}
		{\small Correo electrónico de autor\par}
		
		% Resumen y palabras clave con un pequeño desplazamiento hacia la izquierda
		\hspace{-5cm}{\small \textbf{Abstract.} Resumen hasta 200 palabras. \par}
		\hspace{-5.2cm}{\small \textbf{Keywords:} Máximo 5 palabras claves.\par}
	\end{titlepage}
	
	% Índice
	\tableofcontents
	\thispagestyle{empty}
	\clearpage
	% Índice de figuras
	\listoffigures
	\thispagestyle{empty}
	\clearpage
	
	\section{Desarrollo de la actividad propuesta}
	Aquí desarrollas el contenido principal del trabajo. \parencite{marquez2011metaversos}
	
	% Ejemplo de figura
	\begin{figure}[h]
		\centering
		\includegraphics[width=0.5\textwidth]{uner.png}
		\caption{Descripción de la figura}
		\label{fig:ejemplo}
	\end{figure}
	
	prueba springer  \cite{cheng2023metaverse}
	
	\section{Introducción}
	Esta es la introducción de nuestro documento. Aquí discutimos brevemente el propósito de este trabajo y ofrecemos un panorama general de lo que se discutirá en las siguientes secciones.
	
	\subsection{Contexto histórico}
	Esta es una subsección donde hablamos del contexto histórico de nuestro tema. Es importante entender la historia para poder comprender completamente nuestro tema de estudio.
	
	\subsubsection{Europa en el siglo XIX}
	En esta subsección hablamos de Europa durante el siglo XIX. Durante este período, Europa vio muchos cambios políticos, sociales y tecnológicos que sentaron las bases para el mundo moderno.
	
	El siglo XIX en Europa estuvo marcado por la Revolución Industrial, el surgimiento de los movimientos nacionalistas, y grandes avances en ciencia y tecnología. Además, las potencias europeas expandieron sus imperios coloniales, lo que tuvo un impacto profundo en la geopolítica global.
	
	\subsection{Importancia del estudio}
	Aquí discutimos por qué es importante estudiar nuestro tema. Puede ser porque tiene relevancia actual, porque hay lagunas en la investigación existente, o simplemente porque es un área de interés personal.
	
	\section{Metodología}
	En esta sección, describimos cómo llevamos a cabo nuestra investigación. Esto incluye los métodos que usamos, las fuentes que consultamos y cualquier otro detalle relevante que pueda interesar a nuestros lectores.
	
	\pagebreak
	\section{Conclusiones}
	Presenta las conclusiones del trabajo.
	\pagebreak
	\printbibliography[heading=bibintoc]
\end{document}
