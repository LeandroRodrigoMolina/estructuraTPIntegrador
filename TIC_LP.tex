\documentclass[a4paper,10pt]{article}

% Paquetes requeridos
\usepackage[utf8]{inputenc}
\usepackage[spanish]{babel}
\usepackage{csquotes}
\usepackage{amsmath, amssymb, amsfonts}
\usepackage{graphicx}
\usepackage[style=apa, backend=biber, natbib=true, language=spanish, url=true]{biblatex}
\usepackage{tocloft} % Para personalizar el índice
\usepackage[left=3.5cm,right=2.5cm,top=3.5cm,bottom=3.8cm]{geometry}
\usepackage{setspace} % Espaciado
\usepackage{titlesec} % Para personalizar los títulos
\usepackage{fancyhdr} % Para encabezados y pies de página personalizados

\addbibresource{referencias.bib}
\DeclareLanguageMapping{spanish}{spanish-apa}
% Configuraciones
\setlength{\parskip}{6pt} % Espacio entre párrafos
\setstretch{1.15} % Espacio entre líneas

\renewcommand{\cftsecleader}{\cftdotfill{\cftdotsep}} % Para puntos en el índice

% Estilos para títulos y subtítulos
\titleformat{\section}
{\normalfont\fontsize{12}{15}\bfseries}{\thesection}{1em}{}
\titleformat{\subsection}
{\normalfont\fontsize{11}{14}\bfseries}{\thesubsection}{1em}{}
\titleformat{\subsubsection}
{\normalfont\fontsize{10.5}{13}\bfseries}{\thesubsubsection}{1em}{}

% Configuración de fancyhdr
\pagestyle{fancy}
\fancyhf{} % Limpia encabezados y pies de página
\fancyhead[L]{\leftmark} % Sección en el encabezado izquierdo
\fancyfoot[R]{\thepage} % Número de página en el pie derecho
\renewcommand{\headrulewidth}{0.4pt} % Línea debajo del encabezado
\renewcommand{\footrulewidth}{0pt}   % Sin línea encima del pie

\usepackage[hypertexnames=false]{hyperref}

% Inicio del documento
\begin{document}
% Carátula
\begin{titlepage}
	\centering
	\vspace*{1.5cm}
	\includegraphics[width=0.3\textwidth]{unerlogo.png} % <- Logo de la universidad
	\linebreak
	\vspace{0.5cm} % <- Espacio vertical después del logo
	{\LARGE\bfseries Nombre de la Universidad\par}
	{\Large Facultad de [nombre de la facultad]\par}
	\vspace{0.5cm}
	{\Large Carrera de [nombre de la carrera]\par}
	\vspace{0.5cm}
	{\Large [Nombre del primer alumno]\par}
	{\Large [Nombre del segundo alumno]\par}
	\vfill
	{\large Fecha de presentación: \today\par}
\end{titlepage}
	
	% Índice
	\tableofcontents
	\clearpage
	% Índice de figuras
	\listoffigures
	\clearpage
	
	\section{Desarrollo de la actividad propuesta}
	Aquí desarrollas el contenido principal del trabajo. \parencite{marquez2011metaversos}
	
	% Ejemplo de figura
	\begin{figure}[h]
		\centering
		\includegraphics[width=0.5\textwidth]{uner.png}
		\caption{Descripción de la figura}
		\label{fig:ejemplo}
	\end{figure}
	
	prueba springer  \cite{cheng2023metaverse}
	
	\pagebreak
	\section{Conclusiones}
	Presenta las conclusiones del trabajo.
	\pagebreak
	\printbibliography[heading=bibintoc]
\end{document}
